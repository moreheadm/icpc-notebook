\chapter{Combinatorial}

\section{Permutations}
	\subsection{Factorial}
		\import{factorial.tex}
		%\kactlimport{IntPerm.h}

	\subsection{Cycles}
		Let $g_S(n)$ be the number of $n$-permutations whose cycle lengths all belong to the set $S$. Then
		$$\sum_{n=0} ^\infty g_S(n) \frac{x^n}{n!} = \exp\left(\sum_{n\in S} \frac{x^n} {n} \right)$$

	\subsection{Burnside's lemma}
		Given a group $G$ of symmetries and a set $X$, the number of elements of $X$ \emph{up to symmetry} equals
		 \[ {\frac {1}{|G|}}\sum _{{g\in G}}|X^{g}|, \]
		 where $X^{g}$ are the elements fixed by $g$ ($g.x = x$).

		 If $f(n)$ counts ``configurations'' (of some sort) of length $n$, we can ignore rotational symmetry using $G = \mathbb Z_n$ to get
		 \[ g(n) = \frac 1 n \sum_{k=0}^{n-1}{f(\text{gcd}(n, k))} = \frac 1 n \sum_{k|n}{f(k)\phi(n/k)}. \]

     The number of orbits of a set $X$ under the group action $G$ equals the average number of elements of $X$ fixed by the elements of $G$.\\
     Here’s an example. Consider a square of $2n \times 2n$ cells. How many ways are there to color it into $X$ colors, up to rotations and/or reflections?
     Here, the group has only $8$ elements (rotations by $0, 90, 180, 270$ degrees, reflections over two diagonals, over a vertical line and over a horizontal line).
     Every coloring stays itself after rotating by $0$ degrees, so that rotation has $X^{4n^2}$ fixed points.
     Rotation by 180 degrees and reflections over a horizonal/vertical line split all cells in pairs that must be of the same color
     for a coloring to be unaffected by such rotation/reflection, thus there exist $X^{2n^2}$ such colorings for each of them.
     Rotations by $90$ and $270$ degrees split cells in groups of four, thus yielding $X^n$ fixed colorings.
     Reflections over diagonals split cells into $2n$ groups of $1$ (the diagonal itself) and $2n^2-n$  groups of $2$ (all remaining cells),
     thus yielding $X^{2n^2-n+2n} = X^{2n^2+n}$ affected colorings. So, the answer is: $\frac{X^{4n^2} + 3X^{2n^2} + 2X^{n^2} + 2X^{2n^+n}}{8}$.

\section{Partitions and subsets}
	\subsection{Partition function}
		Number of ways of writing $n$ as a sum of positive integers, disregarding the order of the summands.
		\[ p(0) = 1,\ p(n) = \sum_{k \in \mathbb Z \setminus \{0\}}{(-1)^{k+1} p(n - k(3k-1) / 2)} \]
		\[ p(n) \sim 0.145 / n \cdot \exp(2.56 \sqrt{n}) \]

		\begin{center}
		\begin{tabular}{c|c@{\ }c@{\ }c@{\ }c@{\ }c@{\ }c@{\ }c@{\ }c@{\ }c@{\ }c@{\ }c@{\ }c@{\ }c}
			$n$    & 0 & 1 & 2 & 3 & 4 & 5 & 6  & 7  & 8  & 9  & 20  & 50  & 100 \\ \hline
			$p(n)$ & 1 & 1 & 2 & 3 & 5 & 7 & 11 & 15 & 22 & 30 & 627 & $\mathtt{\sim}$2e5 & $\mathtt{\sim}$2e8 \\
		\end{tabular}
		\end{center}

	\subsection{Binomials}
		\kactlimport{binomialModPrime.h}
		\kactlimport{multinomial.h}

\section{General purpose numbers}
	\subsection{Bernoulli numbers}
		EGF of Bernoulli numbers is $B(t)=\frac{t}{e^t-1}$ (FFT-able).
		$B[0,\ldots] = [1, -\frac{1}{2}, \frac{1}{6}, 0, -\frac{1}{30}, 0, \frac{1}{42}, \ldots]$

		Sums of powers:
		\small
		\[ \sum_{i=1}^n n^m = \frac{1}{m+1} \sum_{k=0}^m \binom{m+1}{k} B_k \cdot (n+1)^{m+1-k} \]
		\normalsize

		Euler-Maclaurin formula for infinite sums:
		\small
		\[ \sum_{i=m}^{\infty} f(i) = \int_m^\infty f(x) dx - \sum_{k=1}^\infty \frac{B_k}{k!}f^{(k-1)}(m) \]
		\[ \approx \int_{m}^\infty f(x)dx + \frac{f(m)}{2} - \frac{f'(m)}{12} + \frac{f'''(m)}{720} + O(f^{(5)}(m)) \]
		\normalsize

	\subsection{Stirling numbers of the first kind}
		Number of permutations on $n$ items with $k$ cycles.
		\begin{align*}
			&c(n,k) = c(n-1,k-1) + (n-1) c(n-1,k),\ c(0,0) = 1 \\
			&\textstyle \sum_{k=0}^n c(n,k)x^k = x(x+1) \dots (x+n-1)
		\end{align*}
		$c(8,k) = 8, 0, 5040, 13068, 13132, 6769, 1960, 322, 28, 1$ \\
		$c(n,2) = 0, 0, 1, 3, 11, 50, 274, 1764, 13068, 109584, \dots$

	\subsection{Eulerian numbers}
		Number of permutations $\pi \in S_n$ in which exactly $k$ elements are greater than the previous element. $k$ $j$:s s.t. $\pi(j)>\pi(j+1)$, $k+1$ $j$:s s.t. $\pi(j)\geq j$, $k$ $j$:s s.t. $\pi(j)>j$.
		$$E(n,k) = (n-k)E(n-1,k-1) + (k+1)E(n-1,k)$$
		$$E(n,0) = E(n,n-1) = 1$$
		$$E(n,k) = \sum_{j=0}^k(-1)^j\binom{n+1}{j}(k+1-j)^n$$

	\subsection{Stirling numbers of the second kind}
		Partitions of $n$ distinct elements into exactly $k$ groups.
		$$S(n,k) = S(n-1,k-1) + k S(n-1,k)$$
		$$S(n,1) = S(n,n) = 1$$
		$$S(n,k) = \frac{1}{k!}\sum_{j=0}^k (-1)^{k-j}\binom{k}{j}j^n$$

	\subsection{Bell numbers}
		Total number of partitions of $n$ distinct elements. $B(n) =$
		$1, 1, 2, 5, 15, 52, 203, 877, 4140, 21147, \dots$. For $p$ prime,
		\[ B(p^m+n)\equiv mB(n)+B(n+1) \pmod{p} \]
    $$B(n) = \sum_{k=0}^n \binom{n}{k}B_k$$

	\subsection{Labeled unrooted trees(Cayley)}
		\# on $n$ vertices: $n^{n-2}$ \\
		\# on $k$ existing trees of size $n_i$: $n_1n_2\cdots n_k n^{k-2}$ \\
		\# with degrees $d_i$: $(n-2)! / ((d_1-1)! \cdots (d_n-1)!)$

	\subsection{Catalan numbers}
		\[ C_n=\frac{1}{n+1}\binom{2n}{n}= \binom{2n}{n}-\binom{2n}{n+1} = \frac{(2n)!}{(n+1)!n!} \]
		\[ C_0=1,\ C_{n+1} = \frac{2(2n+1)}{n+2}C_n,\ C_{n+1}=\sum C_iC_{n-i} \]
		${C_n = 1, 1, 2, 5, 14, 42, 132, 429, 1430, 4862, 16796, 58786, \dots}$
		\begin{itemize}[noitemsep]
			\item sub-diagonal monotone paths in an $n\times n$ grid.
			\item strings with $n$ pairs of parenthesis, correctly nested.
			\item binary trees with with $n+1$ leaves (0 or 2 children).
			\item ordered trees with $n+1$ vertices.
			\item ways a convex polygon with $n+2$ sides can be cut into triangles by connecting vertices with straight lines.
			\item permutations of $[n]$ with no 3-term increasing subseq.
		\end{itemize}
 \section{General purpose theorems - 1}
    \subsection{Identities}
    Vandermonde Convolution: $\binom{m+n}{r} = \sum_{k=0}^r\binom{m}{k}\cdot\binom{r}{n-k}$.\\
    Hockey Stick: $\binom{n+1}{r+1} = \sum_{i=r}^n\binom{i}{r}$.\\

    \subsection{Cycle Lemma}
    Any sequence of $m X$'s and $n Y$'s, where $m > n$ has exactly $m -n$ cyclic permutations which are dominating,
    and $m - kn$ which are $k$-dominating. To find them, arrange sequence in a circle and repeatedly remove
    adjacent pairs $XY$. The remaining $X$'s were each the start of a dominating permutation.

    \subsection{Sprague Grundy theorem}
    Every impartial game is equivalent to a nimber. Nimbers are de-fined inductively as $*0 = \{\}, *1 = {*0}, *2 = {*0, *1}, *(n+1) = *n \cup {n}$, and
    corresponds to a heap of size $n$. The formula for adding positions is $S + S' = {S + s' | s' \in S'} \cup {s + S' | s \in S}$.\\
    $a + b = a \oplus b + 2(a\&b)$.\\
    Define minimum exclusion $M : \phi(N) \rightarrow N$ by $M(S) = $ the least non-negative integer not in S.
    Let $C = (M(A) \oplus B) \cup (M(B) \oplus A)$. Then $M(C) = M(A) \oplus M(B)$.
    Define $SG(S) = M (\{SG(s) | s \in S\})$.
    $SG(Nim_k) = k$ by strong induction. Game is losing iff $SG(S) = 0$. Theorem: $SG(A + B) = SG(A) \oplus SG(B)$.

    \subsection{Partisan Game}
    Can define the negative of a game by interchanging $L$ and $R$'s possible moves.
    Define $G = 0$ if first player loses. $G = H$ if $G + (-H) = 0$.
    A cold game is one which moving only hurts players. In this case we never have G fuzzy 0, so G is
    representable as an integer, thus calculable by DP.

    \subsection{Matrices for operators}
    Matrices for xor, and, and or are: $\frac{1}{\sqrt{2}} \begin{bmatrix} 1 & 1 \\ 1 & -1\end{bmatrix}, \begin{bmatrix} 0 & 1 \\ 1 & 1\end{bmatrix}$
    with inverses: $\begin{bmatrix} -1 & 1 \\ 1 & 0\end{bmatrix}, \begin{bmatrix} 1 & 1 \\ 1 & 0\end{bmatrix}$.
    
 \section{General purpose theorems - 2}

    \subsection{Prufer sequences}
    The set of labeled trees on $n$ vertices corresponds bijectively to the set of Prufer sequences of length $n-2$.
    To convert a tree into a Prufer sequence, repeatedly remove the leaf with the smallest label, and write down its neighbor.
    To convert sequence to tree, first set the degree of each vertex to $n_v + 1$, where $n_v$ is the number of times
    the vertex appears in the sequence. Then for each $i$, find lowest $j$ with degree $1$, add edge $a_i, j$, and decrease
    the degrees of $a_i$ and $j$ by $1$. After this, two nodes of degree 1 remain - connect them.\\
    This can be used to calculate number of labeled trees in a complete bipartite graph - $l^{r-1}\cdot r^{l-1}$.

    \subsection{Tournament Graphs}
    There exists a Hamiltonian path on any tournament graphs - use induction to find. Cycle if strongly connected. TFAE:
    \begin{enumerate}
      \item
      $T$ is transitive.
      \item
      $T$ is strict total ordering.
      \item
      $T$ is acyclic.
      \item
      $T$ has no cycle of length $3$.
      \item
      The outdegrees are $\{0, 1, \cdots, n-1\}$.
      \item
      $T$ has exactly one Hamiltonian path.
    \end{enumerate}
    \subsection{Landau's theorem}
    A sequence of numbers is called a score sequence if for each subset $S$, sum of 
    numbers in $S$ is at least $\binom{|S|}{2}$ and sum of all numbers is $\binom{n}{2}$.\\
    This score sequence represents the outdegrees of a vertex in a tournament graph.

 \section{General purpose theorems - 3}

    \subsection{Dilworth's / Hall's / Mirsky's theorem}
    Maximum antichain has same size as minimum chain decomposition.\\
    Maximum chain size has same size as minimum antichain decomposition.\\
    To compute size, model as bipartite graph with two copies of vertices - $v_in$ and $v_out$.
    Distinct representatives can be chosen for a family of sets $S$ iff every subfamily $W$ of $S$ has at least $|W|$
    elements in their union. E.g. Left side of bipartite graph can be fully matched iff each subset has sufficient ”degree”.

    \subsection{Laplacian Matrix and Kirchoff’s Theorem}

    Laplacian matrix is defined as $L = D - A$, where $D$ is the degree matrix (diagonal), and $A$ the adjacency matrix.\\
    Kirchoff’s Theorem states that the number of spanning trees in a graph is any cofactor of the Laplacian.\\
    To calculate that, remove the first row and column and calculate the determinant of the remaining matrix.

    \subsection{Derangements}
		Permutations of a set such that none of the elements appear in their original position.
    $$D_n = n!\sum_{k=0}^n \frac{(-1)^k}{k!} = (n-1)(D_{n-2}-D_{n-1}) = \left\lfloor\frac{n!}{e}\right\rceil$$




