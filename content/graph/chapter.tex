\chapter{Graph}

%\section{Fundamentals}
%	\kactlimport{BellmanFord.h}
%	\kactlimport{FloydWarshall.h}
% \kactlimport{TopoSort.h}

\section{Euler walk}
	\kactlimport{EulerWalk.h}

\section{Network flow}
	\kactlimport{PushRelabel.h}
	\kactlimport{MinCostMaxFlow.h}
	\kactlimport{EdmondsKarp.h}
	\kactlimport{Dinic.h}
%	\kactlimport{MinCut.h}
	\kactlimport{GlobalMinCut.h}

\section{Matching}
	\kactlimport{hopcroftKarp.h}
	\kactlimport{DFSMatching.h}
	\kactlimport{MinimumVertexCover.h}
	\kactlimport{WeightedMatching.h}
	\kactlimport{GaleShapley.h}
	\kactlimport{GeneralMatching.h}

\section{DFS algorithms}
	%\kactlimport{SCC.h}
	%\kactlimport{BiconnectedComponents.h}
	\kactlimport{ArticulationPointAndBridges.h}
	\kactlimport{2sat.h}

\section{Heuristics}
	\kactlimport{MaximalCliques.h}
	\kactlimport{MaximumClique.h}
	%\kactlimport{MaximumIndependentSet.h}

\section{Trees}
	%\kactlimport{TreePower.h}
	%\kactlimport{LCA.h}
	\kactlimport{CompressTree.h}
	%\kactlimport{HLD.h}
	\kactlimport{LinkCutTree.h}
	\kactlimport{DirectedMST.h}

\section{Math 1}
	\subsection{Number of Spanning Trees}
		% I.e. matrix-tree theorem.
		% Source: https://en.wikipedia.org/wiki/Kirchhoff%27s_theorem
		% Test: stress-tests/graph/matrix-tree.cpp
		Create an $N\times N$ matrix \texttt{mat}, and for each edge $a \rightarrow b \in G$, do
		\texttt{mat[a][b]--, mat[b][b]++} (and \texttt{mat[b][a]--, mat[a][a]++} if $G$ is undirected).
		Remove the $i$th row and column and take the determinant; this yields the number of directed spanning trees rooted at $i$
		(if $G$ is undirected, remove any row/column).

	\subsection{Erdős–Gallai theorem}
		% Source: https://en.wikipedia.org/wiki/Erd%C5%91s%E2%80%93Gallai_theorem
		% Test: stress-tests/graph/matrix-tree.cpp
		A simple graph with node degrees $d_1 \ge \dots \ge d_n$ exists iff $d_1 + \dots + d_n$ is even and for every $k = 1\dots n$,
		\[ \sum _{i=1}^{k}d_{i}\leq k(k-1)+\sum _{i=k+1}^{n}\min(d_{i},k). \]

	\subsection{Konig's theorem}
  In any bipartite graph, the number of edges in a maximum mathing equals the number of vertices in a minimum vertex cover. To exhibit the vertex cover:\\
  \begin{enumerate}
  \item
  Find a maximum matching.
  \item
  Change each edge {\bf used} in the matching into a directed edge from {\bf right to left}.
  \item
  Change each edge {\bf not used} in the matching into a directed edge from {\bf left to right}.
  \item
  Compute the set $T$ of all vertices reachable from unmatched vertices on the left (including themselves).
  \item
  The vertex cover consists of all vertices on the right that are {\bf in} $T$, and all vertices on the left
  that are {\bf not in} $T$.
  \end{enumerate}

\section{Math 2}
	\subsection{Minimum Edge cover}
  If a minimum edge cover contains $C$ edges, and a maximum matching contains $M$ edges, then $C+M = |V|$. To obtain
  the edge cover, start with a maximum matching, and then, for every vertex not matched, just select some edge
  incident upon it and add it to the edge set.

	\subsection{Maximum Independent set}
  To obtain a maximum independent set of a graph, find a max
  clique of the complement. If the graph is bipartite, see MinimumVertexCover.
